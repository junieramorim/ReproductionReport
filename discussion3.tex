As explained in previous sections, the trade-off that there is among the communication overhead and quality results is an important aspect to be considered at the moment of choosing which algorithm to use. The right choice depends on what is the priority. If the allocation needs to be the best according to the sensors available, it is necessary to know that there will be an overhead in communications among the agents. 

In that way, it is necessary to know the network capacity in order to support the messages exchange. Thus, the simulation shows that this aspect is a mandatory requirement to explore the advantages of the modified algorithms. If the communication structure is not good enough or there is no evidence that it supports high volume of messages, it is appropriate reduce the quality result but to keep a certain level of mission accomplishment.

Based on the replications done, a real scenario described by a set of targets to be photographed can be an example of these algorithms application. Thus, it is necessary to choose if it is more important have best pictures with more suitable cameras according to the type of targets, or it is necessary to keep the communication structure not overloaded with high quantity of messages (tokens). This example represents a real possible situation where this trade-off needs to be analyzed properly to obtain the best results according to the demand.
