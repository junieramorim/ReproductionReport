The proposal based on Swarm-GAP presented in this work is divided in three variants. This section describes each of them as well as the motivation behind them.

\subsection{Allocation Loop (AL)} \label{sec:al}

When Swarm-GAP is used to solve the task allocation problem presented in Section \ref{sec:problem}, the resources of the UAVs were not taken fully into account: many tasks were not performed and resources were underused. To take advantage of the available resources of the UAVs and maximize the amount of tasks that are performed, an Allocation Loop (AL) algorithm was proposed. 

Algorithm \ref{algo:swarm-gap-loop} provides the pseudo code of this method.
In AL,instead of dropping the token after all UAVs have received it, the list of visited agents in the token is cleaned and a new token-sending round is started. Thus, each UAV may receive the token more than once and the unallocated tasks will have a new chance of being allocated.
However, a simple cleaning scheme of this list may not be effective. When a token still has unallocated tasks, but the UAVs already allocated all their resources, the token is unnecessarily kept in circulation, thus causing an unnecessary exchange of messages among the UAVs. 

% mudei pq estava errado: estava "the agents that still have capability" mas na verdade são os que NAO tem que são inserido na lsita novamente, pq isso coloquei "do not have" no lugar do still.
To prevent the token remain circulating forever, after cleaning the list of visited agents, the agents that do not have capability or available resources for executing tasks are inserted in this list again. Then, only agents with some probability of allocating tasks will may receive the token in next round. For this, when a \uav\ receives a token, after selecting the tasks and mark itself as visited (lines \ref{line:recebe} to \ref{line:ALini}), it informs the token if it is or is not available to carry out any of the remaining tasks (line \ref{line:informadisponibilidade}). Then, the token stores a list of agents that should no longer receive the token in the next rounds, called list of unavailable agents, because they can no longer select any other task contained in the token.

\begin{algorithm}[h!t]
	\caption{Pseudo code - Allocation loop (AL)}
	\label{algo:swarm-gap-loop}
	
	\SetAlgoLined
	\DontPrintSemicolon
	\SetKwBlock{Loop}{loop}{end loop}
	\SetKwFor{ForAll}{for all}{do}{end for}
	\SetNlSty{text}{}{:}
	\SetNlSkip{0.3em}
	
	Receive Token\; \label{line:recebe}
	
	Select the tasks that it will perform (lines \ref{line:compute_r} to \ref{line:decrease_r} of the Algorithm \ref{algo:swarm-gap}) \;
	
	Mark agent as visited in the token\; \label{line:ALini}
	\If{there are still available tasks}{
		Inform token if it has availability to perform any one of these tasks\; \label{line:informadisponibilidade}
		\If{all agents already receive the token}{
			Clean the list of visited agents\;
			Fill list of visited agents with the unavailable agents\;
		}
		Send the token to a not yet visited agent\; \label{line:ALfim}
	}
\end{algorithm}


\subsection{Sorting and Allocation Loop (SAL)} \label{sec:sal}

In the previous AL algorithm, since the choice of tasks is done in the order that the tasks are in the token, often the \uavs\ end up selecting first the tasks
that are not those more suitable for them. Then, it may cause lack of resource for others tasks more appropriate for them. To deal with this problem, a sorting mechanism that complement the AL is proposed. This algorithm is called Sorting and Allocation Loop (SAL).

The SAL's pseudo code is presented in Algorithm \ref{algo:sal}.
In SAL, the tasks are sorted by tendency in decreasing order (line \ref{line:forallALA} to \ref{line:sortbyTend}). This facilitates the selection of tasks by the \uavs, i.e., it makes easier to the \uavs\ select the more appropriate tasks to them. This sort gives priority to the tasks with the greatest tendency, avoiding the selection of tasks with lower tendency.

\begin{algorithm}[h!t]
	\caption{Pseudo code - SAL}
	\label{algo:sal}
	
	\SetAlgoLined
	\DontPrintSemicolon
	\SetKwBlock{Loop}{loop}{end loop}
	\SetKwFor{ForAll}{for all}{do}{end for}
	
	%\SetAlgoNlRelativeSize{-3}
	\SetNlSty{text}{}{:}
	\SetNlSkip{0.3em}
	
	Receive Token\;
	Compute available resources $r_i $ \; \label{line:compute_rALA}
	
	\ForAll{ available tasks }{ \label{line:forallALA}
		Compute capability $k_{ij}$\; \label{line:compute_kALA}
		Compute tendency $T_{\theta_{ij}}$ \;  \label{line:compute_tALA}
	}
	Sort tasks by descending tendency\; \label{line:sortbyTend}
	
	\ForAll{ available tasks sorted by tendency }{
		The same as lines \ref{line:ini_ifalgo1} to \ref{line:fim_ifalgo1} of the Algorithm \ref{algo:swarm-gap}

	}
	
	The same as lines \ref{line:ALini} to \ref{line:ALfim} of the Algorithm \ref{algo:swarm-gap-loop} \;
	
\end{algorithm}

\subsection{Limit and Allocation Loop (LAL)} \label{sec:lal}

By the way it was conceived, the SAL algorithm causes an increase in the amount of performed tasks. Nevertheless, it was observed in the experiments that, in some runs, there were UAVs that still remained idle, i.e, UAVs that performed no task. When the UAVs have enough resource, the first one visited by the token allocates most of the tasks. Therefore, it may result in idleness for other UAVs. 

For this reason, the Limit and Allocation Loop (LAL) algorithm adds to SAL a selection limit per UAV in each round. This limit means that a \uav\ cannot allocate more than one task each time it receives the token. Thus, all \uavs\ have the chance to choose a task within a round and the workload is in result more balanced. 
%\linebreak

% Atendendo ao revisor 2:
Nevertheless, since the method is probabilistic, there is no guarantee that all agents will select some task. The method does not enforce an agent to choose a task; it only limits the number of tasks that are selected per round. Therefore, there may still be cases with idle agents. However, in general, the workload balance improves. Moreover, cases in which an optimal allocation of tasks is achieved may happen even when there are some idle agents.