The authors in \cite{ferreira2007swarm} have proposed the Swarm-GAP, an algorithm for distributed task allocation based on theoretical models of division of labor in social insect colonies. As the work here proposed is based on the Swarm-GAP, this section briefly presents an overview of this algorithm.

Swarm-GAP allows that each agent chooses which tasks it will perform. This decision is based just on local information, thus leading to little communication the agents. The task allocation is modeled as a generalized assignment problem (GAP) and the goal is to maximize the total capability.
Swarm-GAP is a probabilistic approach, by using the mathematical response threshold model formulated by Theraulaz et al. in \cite{theraulaz1998response}.

% Atendendo ao revisor 2:
The response threshold $\textrm{T}_{\theta_{ij}}$ (Equation \ref{eq:tendencia}) expresses the likelihood of an agent to react to task-associated stimuli $st_{j}$ and thus to execute it. The threshold $\theta_{ij}$ is based on the agent's capability to perform a task (Equation \ref{eq:limiar}). The higher the capacity, the lower the threshold. 
In case in which the task has a low-stimulus, it is performed by specialized agents. As the stimulus of a task increases (and hence the just mentioned likelihood), less specialized agents also start performing the task \cite{bonabeau1999swarm}.

\begin{equation} \label{eq:tendencia}
	\textrm{T}_{\theta_{ij}}(st_j) = \frac{st_{j}^2}{st_{j}^2 + \theta_{ij}^2}
\end{equation}

\begin{equation} \label{eq:limiar}
	\theta_{ij} = 1 - k_{ij}
\end{equation}

Swarm-GAP uses a communication model based on a token passing protocol, as can be observed in its pseudo code presented in Algorithm \ref{algo:swarm-gap}. Once an agent receives a token (line \ref{line:receivetoken}), it decides which tasks it will perform (line \ref{line:forall} to \ref{line:fim_ifalgo1}). The agent likelihood to choose a task is determined by the tendency $\textrm{T}_{\theta_{ij}}$ (Equation \ref{eq:tendencia}). 
This tendency is calculated with the task's stimulus $st_{j}$ and threshold $\theta_{ij}$.

The decision also depends on the available resources (line \ref{line:ini_ifalgo1}), i.e., the agent must have enough resource to perform the task. 
When the agent decides to carry out a task, this task is allocated to the agent (line \ref{line:aloca_sgap}) and the agent's resources are reduced accordingly (line \ref{line:decrease_r}). After that, the agent is marked as visited (\ref{line:marcavisitado}) and if there are still unallocated tasks, the agent sends the token to an agent which has not yet received that specific token (line \ref{line:ifstillavatask} to \ref{line:envia}).

\begin{algorithm}[!ht]
	\SetAlgoLined
	\DontPrintSemicolon
	\SetKwBlock{Loop}{loop}{end loop}
	\SetKwFor{ForAll}{for all}{do}{end for}
	\SetNlSty{text}{}{:}
	\SetNlSkip{0.3em}	
	
	\caption{Pseudo code of the Swarm-GAP }
	\label{algo:swarm-gap}
	Receive Token\; \label{line:receivetoken}
	Compute available resources $r_i $ \; \label{line:compute_r}
	\ForAll{ available tasks }{ \label{line:forall}
		Compute capability $k_{ij}$\; \label{line:compute_k}
		Compute tendency $T_{\theta_{ij}(st)}$ \;  \label{line:compute_t}
		\If{ roulette() $< T_{\theta_{ij}(st)}$ and $r_i \geq c_j $}{ \label{line:ini_ifalgo1}
			Allocate task $j$ to agent $i$ \; \label{line:aloca_sgap}
			Decrease resource $r_i $\; \label{line:decrease_r} \label{line:fim_ifalgo1}
		}
	}
	Mark agent as visited in the token\; \label{line:marcavisitado}
	\If{there are still available tasks}{ \label{line:ifstillavatask}
		Send the token to a not yet visited agent\; \label{line:envia}
	}
\end{algorithm}

Swarm-GAP also allows that an agent creates a token when it perceives a task in the environment that needs to be done. This feature is not further discussed here because, in the context of this work, the token with the tasks is only created by a central entity.  For more details about Swarm-GAP, see \cite{ferreira2007swarm} or \cite{ferreira2010robocup}.