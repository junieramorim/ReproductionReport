This paper has proposed a solution to the task allocation problem in a team of UAVs in a decentralized way. The tackled problem assumes that the tasks are created by a central entity, such as in several military operations, and refers to the activity of monitoring an area with the objective of detecting certain types of targets. 

The proposed solution was developed from the Swarm-GAP algorithm, which uses a swarm intelligence approach based on response threshold model. Through experiments observations, features were identified that could promote better allocation of tasks, and consequently increasing the total reward. The features were used to evolve the solution, resulting in three algorithm variants: Allocation Loop (AL), Sorting and Allocation Loop (SAL) and Limit and Allocation Loop (LAL). These variants were evaluated, presenting positive results compared to Swarm-GAP.

In the performed experiments, it was assumed that communication among the UAVs is complete and never fails. However, this is a strong assumption. Thus, an important future work is to consider fault tolerance. Another issue to be addressed  is the dependency among tasks and teamwork (e.g., when two or more agents are necessary to perform the same task). 

% Para atender ao revisor 1:
Experiments with dynamic teams will also be conducted in the future. Although the experiments were performed with non-dynamic teams, the proposed method is flexible enough to support also dynamic assignment of the UAVs to the teams, which is a promising direction for future work. This is possible because agents do not have information on the overall status of the team or information about any other agent. Thus, agents may leave or join the team without influencing the decision-making process.