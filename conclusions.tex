Since the original work reported in \cite{MAS07} is classified as a reproducible research\cite{exp02} and confirmed by the reproduction displayed in Section \ref{sec:background}, we performed two independent replications, using NetLogo 5.3.1 environment, to collect results and obtain evidences about limitations and improvements opportunities proposed by the present study.

With an approaching closer to real world, a dynamic scenario was defined in substitution of the original static one used in \cite{MAS07}. Moreover, a first replication was performed to validate that the original algorithms proposed by \textit{Schwarzrock et al.} are fully functional in this new dynamic scenario. This experiment showed an improvement opportunity due to a decrease in some dependent variables, e.g., capability and quality. 

Proceeding an action research and based on results obtained by the original work, extended algorithms were proposed to better address the new created dynamic scenario. The second replication, assessed these extended algorithms and allowed the identification and discussion of emerging trade-offs. The metrics of quality and exchanged messages were the focus of these analysis due to the variance presented by the extended algorithms in these measurements compared with the original ones.

The proposal here presented has, as the main idea, that all tasks releasing after the team loose an agent, and the token resetting to allow a new turn of allocation execution steps. Thus, the tasks not completed can be reallocated always looking for the quality maximization based on the relation among the task distance and the sensor suitability to its performance. There are evidences that this new procedure requires more communication since the number of exchanged messages increased significantly face the original proposal.

The main discussion is the collateral effect caused to obtain an increased quality. This effect is an increase in the exchanged messages that suggested the idea about the need of a network requirement level to apply the proposed algorithms. However, more detailed measurements of network level requirements are left as suggestion of future work. Furthermore, the used communication structure, initially a token ring network, can impact in results and different topologies can be tested in the future.