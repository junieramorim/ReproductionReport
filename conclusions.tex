Since the original work \cite{MAS07} is classified as a reproducible research\cite{exp02} and confirmed by the reproduction displayed in Section \ref{sec:background}, we performed two independent replications, using NetLogo 5.3.1 environment, to collect results and obtain evidences about limitations and improvements opportunities proposed by the present study.

With an approaching closer to real world, we defined a dynamic context in substitution of the original static one used in \cite{MAS07}, and performed a first replication to validate that the original algorithms proposed by \textit{Schwarzrock et al.} are fully functional in this new dynamic scenario. This experiment showed an improvement opportunity due to a decrease in some dependent variables, e.g., capability and quality. 

Proceeding an action research and based on results obtained by the original work, we proposed extended algorithms to better address the new dynamic scenario created. The second replication, assessed these extended algorithms and permitted to identify and discuss emerging trade-offs. The variables quality and exchanged messages were the focus of these analysis due to the variance presented by the extended algorithms in these variables compared with the original ones.

The proposal here presented has, as the main idea, all tasks releasing after the team loose an agent, and the token resetting to permit a new turn of allocation execution steps. Thus, the tasks not completed can be reallocated always looking for the quality maximization based on the relation among the task distance and the sensor suitability to its realization. There are evidences that this new procedure execution requires more communication since the number of exchanged messages increased more than 100\%.

The main discussion is the collateral effect caused to obtain a quality increasing. This effect is an increase in the exchanged messages that suggested some idea of network requirement level to apply the algorithms proposed. However, this measure of network level requirements is left as suggestion of future work. Furthermore, the communication structure used, initially a ring network, can impact in results and different typologies can be tested in the future.
