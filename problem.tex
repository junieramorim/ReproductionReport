% -------------------------- Para atender ao revisor 1
In the problem addressed in this work, the central command unit (henceforth referred simply as central) creates the missions and sends them to the teams of UAVs, but it does not control the allocation of tasks within the teams. There is no hierarchy among the UAVs in a team and there is no hierarchy among the teams. In order to allow teams to handle several types of tasks, such teams are formed as heterogeneous as possible. However, it is important to highlight that our proposal is to assign tasks to UAVs that are already deployed, i.e. UAVs that are already part of a team operating over a given region. The formation of the teams as well as the elaboration of the missions, represent other problems that are not in this scope, but they can be handled in future complementary work.

Each team operates in a particular region, i.e. the teams are delimited by the region in which they operate. The missions have to be performed in a given region of interest, and the assignment of the team that will perform them is done by the proximity between the team and the mission location. The allocation of the tasks is done within each team and there is no influence or interference of one team in another regarding the task allocation. Hence, the problem addressed in this work refers to the task allocation inside a single team. A non-dynamic team formation is here considered, although in real situations it is possible to consider that the UAVs of a team may change (a discussion about this aspect is presented in Section \ref{sec:conclusions}).
% --------------------------

Each task of a mission created by the central refers to the activity of monitoring an area with the objective of detecting some type of target. Each type of target can be detected by one or more sensors that a \uav\ has. However, each kind of sensor has a different quality in detecting the targets, i.e., some sensors are more suitable for some types of targets than others.

Furthermore, each UAV can be equipped with one or more sensors. Thus, each UAV can perform more than one type of task. However, it is desirable that the tasks be performed by the UAVs that are more suitable for them, that is, those equipped with sensors that offer higher quality to detect the type of target required by the task. Each task needs be performed by only one \uav, but each \uav\ can perform any or many tasks, as long as it has quality and resource to execute them. 

The central has information about the tasks that need to be performed, but does not necessary knows the current status and positioning of the agents in the environment. As mentioned, a mission contains many tasks. It may happen that the \uavs\ cannot complete the whole mission, depending on the quantity of able agents and the time that is needed to execute the set of tasks that composes a mission. 
In this case, the central can send reinforcements to that region, according to the demand of remaining tasks, but this is already another problem which is beyond the scope of this work.
% isso foi para atender o revisor 2. 

The goal here is to find, in a decentralized way, an appropriate task allocation among \uavs\
so that the quantity and quality of the performed tasks is maximized and the time spent on it is minimized. This problem can be modeled as a Generalized Assignment Problem (GAP), that is known to be a NP-complete problem \cite{shmoys1993approximation}. The problem formulation is given as follows.

Let 
$\mathcal{I} = \{i_1, ..., i_m\}$ be a set of $m$ \uavs\   and
$\mathcal{S} = \{s_1, ..., s_u\}$ be a set of $u$ types of sensors.
Each \uav\ $i \in \mathcal{I}$ is modeled by a set of attributes $D_i = \{l, Si, r\}$, where 
$l$ is the coordinates $\langle x,y \rangle$ of the \uav's current location; $Si \subseteq \mathcal{S}$ is a set of types sensors with the \uav\ is equipped and $r$ is the available resource.

A mission $M$ is composed by a set $\mathcal{J} = \{j_1, ..., j_n\}$ of $n$ tasks and it has a deadline $dl$, which determines the maximum time that \uavs\ have to complete the mission. Each task $j \in \mathcal{J}$
is modeled by a set of attributes $E_j = \{l, t, c\}$, where $l$ is the coordinates $\langle x,y \rangle$ of the task's location; $t \in \mathcal{A} = \{a_1, ..., a_v\}$ is the type of target that needs to be detected; and $c$ the amount of resource necessary to survey all task's area. 

The matrix $QM = (q_{sa})_{u \times v}$ with $q_{sa} \in [0,1]$
, is called quality matrix, where the $(s,a)^{th}$ entry $q_{sa}$ corresponds to the sensor $s$' quality to detected the type of target $a$. 
The quality of the \uav\ $i$ to perform a task $j$, is given by Eq. \ref{eq:quality}. When $Q(i,j) = 0$ the \uav\ $i$ no have sensor capable of detecting $t_j$.

\begin{equation} \label{eq:quality}
	Q(i,j) = \max_{s \in Si_i}  q_{st_j}
\end{equation}

Each agent $i$ has a specific capability $k_{ij}$ to perform each task $j$. The capability in this problem is defined by Eq. \ref{eq:capability}, where 
$J$ is the set of available tasks, $d(i,j)$ is the Euclidean distance between the UAV and the task, $Q(i,j)$ is the UAV's quality to perform the task and $\alpha \in [0,1]$  is the weight given to the distance and quality factors.

\begin{equation} \label{eq:capability}
\begin{split}
k_{ij} = \frac{\max_{g \in J} \{d(i,g)\} - d(i,j)}{\max_{g \in J} \{d(i,g)\}} \times \alpha + \\
(1 - \frac{\max_{g \in J} \{Q(i,g)\} - Q(i,j)}{\max_{g \in J} \{Q(i,g)\}}) \times (1-\alpha)
\end{split}
\end{equation}

According to Eq. \ref{eq:capability}, the closer to a task location the UAV is, and the greater its sensors' quality to detect the type of target required by that task, the greater its capability.

The matrix $X = (x_{ij})_{m \times n}$, called allocation matrix, represents the task allocation among \uavs. Where, $x_{ij}$ is 1 if the task $j$ is allocated to \uav\ $i$ and $0$ otherwise. 
The goal is find a allocation matrix $X$ that maximize the total reward, which is given by agent's capabilities (Eq. \ref{eq:objetivo}),

\begin{equation} \label{eq:objetivo}
X = \argmax_{X'} \sum_{i \in \mathcal{I}}\sum_{j \in \mathcal{J}} k_{ij} \times x'_{ij}
\end{equation}

subject to constraints:

\begin{equation} \label{eq:umagenteparacadatarefa}
\forall j \in \mathcal{J} \sum_{i \in \mathcal{I}} x_{ij} \leq 1
\end{equation}

\begin{equation} \label{eg:constQualidade}
\forall x \in X \mid x_{ij} = 1, Q(i,j) > 0 
\end{equation}

\begin{equation} \label{eq:respeitar_recursos}
\forall i \in \mathcal{I} \sum_{j \in \mathcal{J}} x_{ij} \times C_{ij} \leq r_i
\end{equation}

The constraint in Eq. \ref{eq:umagenteparacadatarefa} ensures that each task is allocated to at most one \uav. Eq. \ref{eg:constQualidade} restricts the allocation of the tasks only to the agents that have the quality to execute them. 
Finally, the constraint in Eq. \ref{eq:respeitar_recursos} requires that the \uav's resources limitations must be respected.

For simplicity, a single type of resource is being considered: time. Then, the resources $r_i$ for all \uav\ $i$ is the same as the mission deadline $dl_M$. When a task $j$ is performed by \uav\ $i$, it consumes $C_{ij}$ units of $r_i$. $C_{ij} = d(i,j) + c_j$, where $d(i,j)$ is the distance traveled up by \uav\ $i$ to the task $j$ and $c_j$ is the time to execute the task.
