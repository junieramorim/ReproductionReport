The extended algorithms proposed by this work caused more than 100\% of increase in the number of exchanged messages. This result suggests an existence of network requirements to support the communication necessary. Based on this, it is necessary to detail these requirements to guarantee the possibility of plan which algorithm will be chosen according to the network resources available. Some simulations with the algorithms proposed can provide evidence about the relations among communication restrictions and the quality level desired in the mission accomplishment.

The configuration used by the simulation as a communication structure was a ring network. It would be necessary to assess the impact of using the new algorithms proposed applied on another type of structure, e.g., in a fully connected network. Maybe another architecture can be more compatible and support a higher demand of message exchanging.

Another opportunity to improvement is model and implement a dynamic scenario with more elements changeable. In this work, we are changing only the number of agents, but in a real scenario it could have mission/tasks or agents capabilities changes in runtime. With this new context, we can have another kind of impact that can prevent the use of the algorithms proposed or even the Swarm GAP strategy.

These are situations opened to future works and empirical experiments to evaluate results and create a correlation among network structure and algorithm applied to solve the task allocation problem with a scenario plenty of changeable elements and adopting Swarm-GAP strategy.
