The extended algorithms proposed in this work caused more than 100\% of increase in the number of exchanged messages. This result suggests an existence of network requirements to support the necessary communication. Based on this, it is necessary to detail these requirements to guarantee the possibility of plan which algorithm will be chosen according to the available network resources. Some simulations with the proposed algorithms provide evidence about the relations among communication constraints and the quality level desired in the mission accomplishment.

The configuration used by the simulation as a communication structure was a ring network. It would be necessary to assess the impact of using the new proposed algorithms applied on another type of network structure, e.g., in a fully connected network. Maybe another architecture can be more compatible and support a higher demand of message exchanging.

Another opportunity to improvement is to model and to implement a scenario with more dynamic elements. In this work, only the number of agents were changed, but in a real military operation scenario it could have mission/tasks or agents capabilities changing during runtime. With this new context, other impact may be observed and may required variations of the proposed algorithms or even the original Swarm GAP strategy.

These are situations opened as prospective directions for future investigations. Empirical experiments have to be performed to evaluate correlations among the network structure and the algorithm applied to solve the task allocation problem with a scenario plenty of dynamic elements and adopting Swarm-GAP strategy and its variations.
