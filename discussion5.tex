The extended algorithms proposed in this work caused more than 100\% of increase in the number of exchanged messages. This result assumes an existing network to support the necessary communication, which might not always be available. 

Additionally, the configuration used by the simulation as a communication structure was a ring network. To comply with established Command and Control (C2) strategy models, e.g., Network Enabled Capability (NEC) \cite{CC01}, it would be necessary to assess the impact of using the new proposed algorithms applied on another type of network structure, e.g., in a fully connected network. Maybe another architecture can be more compatible and support a higher demand of message exchange.

Another improvement opportunity is to model and to implement a scenario with more dynamic elements. In this work, only the number of agents was changed, but in a real military operation scenario it could have mission/tasks or agents capabilities changing during runtime. With this new context, additional impact may be observed and require variations of the proposed algorithms or even in the original Swarm GAP strategy.

These are situations opened as prospective directions for future investigations. Empirical experiments have to be performed to evaluate correlations among the network structure and the algorithm applied to solve the task allocation problem with a scenario plenty of dynamic elements and adopting Swarm-GAP strategy and its variations.
