The replications performed with the original and the extended algorithms, in the dynamic scenario, suggested higher results difference in quality and number of messages sent metrics. These highlighted differences justify the choose of these two variables to analyze the algorithms behaviour and the trade-off emerging, in that new dynamic scenario. The messages exchanged represent the token being sent to the next agent in the communication network.

The quality of the sensors related to the tasks guarantees that the agent is the most suitable one to perform it. This condition increases the obtained results even with a reduced number of performed tasks. It can be observed in Figure \ref{fig:fig03} that even the original algorithms working in the dynamic scenario, its quality decreases significantly. 

As the first contribution of this work, it was proved that the original algorithms proposed in \cite{MAS07}, and based on Swarm-GAP intelligence, work in dynamic scenarios. However, the results obtained for all variables were not in the same level of what were obtained by the original study in a static context. All independent variables presented similar values compared with the original study \cite{MAS07}, except the following:

\begin{itemize}
   \item Quality: when the UAVs are disabled (representing a shut down aircraft), the tasks allocated to it are not performed and the quality sum of completed tasks decreased ($\approx 40\%$) as there are fewer number of agents.
   \item Capability: as the agent capability is calculated using the sensors quality,  and it decreases as explained above, this variable suffers a decreasing of $\approx 40\%$
   \item Number of finished tasks: having less UAVs to perform the tasks, it is natural that the number of finished tasks decreases. This variation was $\approx 50\%$ in most of the cases.
   \item Elapsed Time: the time spent to perform the mission was shorter due to the fewer number of UAVs. The remaining UAVs always use the maximum possible available resources to perform tasks. However, in general these capacities finished before the total time available.  Thereby, the elapsed time reduced in $\approx 15\%$, specially in LAL and SAL algorithms. The results in those algorithms where out of the standard deviation compared with the original experiment graphs.
\end{itemize}

Even with the variation above, this solution can be applied in case in which there is no other strategy available to deal with a dynamic scenario. However, there are evidence indicating possibilities of algorithms improvement.

The second contribution is the improvement proposed to those algorithms to deal better with dynamic scenarios. The extended algorithms presented in Section \ref{sec:changes} was submitted to an independent replication and the results were similar to the original algorithms applied in dynamic scenarios except by following metrics:

\begin{itemize}
   \item Quality: this metric presented a significant result recovery ($\approx +40\%$) demonstrating a level equals to the original study applied to a static scenario (see Section \ref{sec:replication}).
   \item Number of exchanged messages: the extended algorithms require more messages transmission due to the token reset. This characteristic causes an increasing of $100\%$ in the number of exchanged messages compared with the original algorithms in dynamic scenario (see Section \ref{sec:replication}).
   \item Number of finished tasks: the number of finished tasks reduced up to $50\%$, compared with the original algorithms, due to the elapsed time to resend tokens when something in the context changes, and the fewer number of available UAVs in some experiments.
\end{itemize}

It is possible to notice that even with a reduced number of finished tasks, the total quality increased due to the tasks reallocation that occurs when a UAV is shut down. This operation always occurs getting the maximum possible quality level based on the available resources, associating the most suitable sensors to the tasks.

Thus, it is possible to increase quality results if the network aspects are not an issue and the communication structure can support the demand required to execute the extended algorithms. It was also possible to identify a message traffic 3 times higher than the original one due to the necessity of communication among the agents using the extended algorithms, that it evidences the network capability requirement.

In summary, evidences were collected, with this empirical study, that there is a trade-off among the quality and the number of exchanged messages when the extended algorithms here proposed are used in dynamic scenarios as described in Section \ref{sec:dynamic_scenario}. The scope of this study was relevant because the dynamic created scenario use assumptions closer to the real world operation scenario, presenting a more realistic (and useful) way to perform tasks allocation.