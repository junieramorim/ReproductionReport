The replications performed with the original and the extended algorithms, in the dynamic scenario, suggested higher results difference in quality and number of messages sent metrics. These highlighted differences justify choosing these two variables to analyze the algorithms behaviour and the emerging trade-off, in that new dynamic scenario. The messages exchanged represent the token being sent to the next agent in the communication network.

The quality of the sensors related to the tasks guarantees that the agent is the most suitable one to perform it. Figure \ref{fig:fig03} shows a significant quality results decrease with the original algorithms in the dynamic scenario. On the other hand, Figure \ref{fig:fig05} shows that extended algorithms increased this variable results, even with a reduced number of performed tasks, bringing the level to the same original baseline. 

As the first contribution, this work provides evidence that the original algorithms proposed in~\cite{MAS07}, and based on Swarm-GAP intelligence, work in dynamic scenarios. However, the results obtained for all variables were not in the same level of what were obtained by the original study in a static context. All dependent variables presented similar values compared with the original study~\cite{MAS07}, except the following:

\begin{itemize}
   \item \textbf{Quality}: when the UAVs are disabled (representing a shut down aircraft), the tasks allocated to it are not performed and the quality sum of completed tasks decreased ($\approx 40\%$) as there are fewer agents;
   \item \textbf{Capability}: as the agent capability is calculated using the sensors quality,  and it decreases as explained above, this variable decreases by $\approx 40\%$;
   \item \textbf{Number of finished tasks}: having fewer UAVs to perform the tasks, it is natural that the number of finished tasks decreases. This variation was $\approx 50\%$ in most cases.
   \item \textbf{Elapsed Time}: the time spent to perform the mission was shorter due to the fewer number of UAVs. The remaining UAVs always use the maximum possible available resource to perform the tasks. However, in general such resource exhausted before the total available time. Thereby, the elapsed time reduced in $\approx 15\%$, particularly in LAL and SAL algorithms. The results in those algorithms fall off the standard deviation compared with the original experimental results.
\end{itemize}

Even with the variation above, this solution can be applied in case in which there is no other strategy available to deal with a dynamic scenario. However, there is evidence indicating possibility of algorithms improvement.

In this vein, the second contribution is the improvement proposed to those algorithms to deal better with the proposed dynamic scenario. The extended algorithms presented in Section \ref{sec:changes} was submitted to an independent replication and the results were similar to the original algorithms applied in dynamic scenarios except for following metrics:

\begin{itemize}
   \item \textbf{Quality}: this metric presented a significant result recovery ($\approx +40\%$) showing a level equal to the original study, which was carried out in a static scenario (Section \ref{sec:replication});
   \item \textbf{Number of exchanged messages}: the extended algorithms require more exchanged messages due to the token reset. This characteristic causes an increase of $100\%$ in the number of exchanged messages compared to the original algorithms in the dynamic scenario (Section \ref{sec:replication});
   \item \textbf{Number of finished tasks}: the number of finished tasks reduced up to $50\%$, compared to the original algorithms in the static scenario, due to the required time to resend tokens when something in the dynamic context occurs, i.e., when there is a an UAV removal. With this, there is a fewer number of available UAVs in the experiments.
\end{itemize}

It is possible to notice that, even with a reduced number of finished tasks, the total quality increased due to the tasks reallocation that occurs when a UAV is shut down. This operation always occurs getting the maximum possible quality level based on the available resources, associating the most suitable sensors to the tasks.

Thus, it is possible to increase quality results if the network aspects are not an issue and the communication structure can support the demand required to execute the extended algorithms. Indeed, it was also possible to identify a message traffic three times higher than the original one due to the communication need among the agents using the extended algorithms, which provides evidence for the network capability requirement.

In summary, this empirical work provides evidence that there is a trade-off among the quality and the number of exchanged messages when the extended algorithms here proposed are used in dynamic scenarios as described in Section~\ref{sec:dynamic_scenario}. The scope of this study is relevant because the dynamic created scenario use assumptions closer to the real world operation scenario, presenting a more realistic (and useful) way to perform tasks allocation.