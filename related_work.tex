Many researchers from the area of multi-agent systems have made efforts to address task allocation problems, hence several approaches have been proposed. This section discusses some of these proposal that deal with this problem in different domains as well as in multi-UAV systems. The focus will be on decentralized approaches. Although there are also centralized solutions, such as \cite{jose2016task} which uses genetic algorithm to assign tasks in a centralized multi-robots system for industrial plant inspection, such solutions are not suitable for the problem addressed in this current work.

A number of proposals employed the threshold-based approach to solve task allocation problems, such as \cite{scerri2005allocatingLADCOP,ferreira2010robocup,ikemoto2010adaptive} and Swarm-GAP \cite{ferreira2007swarm}. The authors in \cite{scerri2005allocatingLADCOP} proposed the LA-DCOP, an distributed threshold-based algorithm to task allocation in rescue scenarios with extreme teams of agents. In \cite{ferreira2010robocup} the authors showed that Swarm-GAP \cite{ferreira2007swarm} and LA-DCOP \cite{scerri2005allocatingLADCOP} had similar results in a RoboCup Rescue scenario with three kinds of agents. In \cite{ikemoto2010adaptive}, the threshold-based approach is applied to food forage labor by a group of robots.

The approach used in this paper is the threshold-based as in \cite{ferreira2007swarm,scerri2005allocatingLADCOP,ferreira2010robocup,ikemoto2010adaptive} and the problem presented here also deals with extreme teams as in \cite{scerri2005allocatingLADCOP, ferreira2010robocup} due to the different types of sensors that the \uavs\ have. However, the way that the agents receive the tasks is different. The tasks in the addressed problem are ordered by central entity, due to strategic military purposes. This generates a greater amount of tasks received by the agents in a single time compared to the assumption that the agents perceive the tasks in the environment as in \cite{ferreira2007swarm,scerri2005allocatingLADCOP, ferreira2010robocup,ikemoto2010adaptive}. It causes drawbacks as discussed in Section \ref{sec:introduction}. However, it is worth to highlight that the perception and creation of tasks can, if necessary, be easily added to our method, since it is based on Swarm-GAP that has this functionality.

Market-based approach also have been widely used to deal with the task allocation problem in different domains. Examples of this approach are those works employing auctions in their methods, as such  \cite{lemaire2004distributed,landen2010complex,ibri2012multi,tolmidis2013multi}. The authors in \cite{lemaire2004distributed,landen2010complex} have applied the task allocation problem in multi-UAV systems as this work. In \cite{lemaire2004distributed} the authors have stated the problem as a TSP and proposed an algorithm based on the Contract-Net protocol. A token-ring based approach is also used to avoid several auctions being launched at the same time. In \cite{landen2010complex}, the aim is to solve complex task allocation among \uavs. To deal with complex tasks the authors have proposed a tree-shaped modeling for the problem, allowing to express the hierarchy of dependence among the tasks. The authors in \cite{ibri2012multi} used auction to assign task to emergency vehicles. Unlike our work, the fleet is composed of homogeneous vehicles. In \cite{tolmidis2013multi} the authors proposed a general task allocation solution for multi-robot systems, begin independent of the domain. The solution works with multi-objective optimization. The authors consider their method ``neither strictly centralized, nor decentralized'' \cite{tolmidis2013multi}.

The methods proposed by \cite{lemaire2004distributed,landen2010complex,ibri2012multi,tolmidis2013multi} are market-based being differently to the one here presented, which uses threshold-based and token-passing approaches, and being also biologically inspired. Comparisons among different approaches to solve the task allocation problem can also be found in \cite{kalra2006comparative,xu2006comparing}. These studies compare market-based, threshold-based and token-passing based approaches, providing insights of their benefits and drawbacks considering different scenarios. Furthermore, in \cite{lemaire2004distributed,landen2010complex,tolmidis2013multi} the agents are also responsible for perceiving the tasks in the environment as in \cite{ferreira2007swarm,scerri2005allocatingLADCOP, ferreira2010robocup,ikemoto2010adaptive}. Once an agent perceives a task, it launches the task to the other agents through auction. 

There are also other works that propose methods for allocating tasks, such as \cite{iijima2016analysis,iijima2017adaptive,branisso2013multi,brutschy2014self}. The authors in \cite{iijima2016analysis,iijima2017adaptive} deal with task allocation in a distributed environment such as the Internet. This assignment is based on the preferences declared by the agents, being closer to an auction-based approach. In \cite{branisso2013multi} the authors propose a solution to task allocation in the domain of material handling in warehouses. In this work, tasks should be performed by Automated Guided Vehicles (AGVs). Their solution uses a fuzzy inference system to support decision making. Like our proposal, in their approach, each agent decides by itself which task it will perform. An agent makes the decision based on its location, loading and storage point information. In this problem, the AGVs are homogeneous agents that have to deal with two types of tasks: that are related to loading and storage. On the other hand, the proposal presented in this paper addresses heterogeneous agents (UAVs), which aim to handle a number of different types of tasks. The authors in \cite{brutschy2014self} deal with tasks that are sequentially interdependent in a multi-robot system. Each agent also decides by itself if it performs or not  a task. An agent uses its perception of the delay experienced when waiting for other agents working on the other subtask(s) to take a decision. The method requires no communication between agents. The tasks in their problem can be perceived in the environment, unlike the tasks referred to the current paper, which are given by a central entity. In summary, despite the similarities in some aspects, the proposals in \cite{branisso2013multi,brutschy2014self} have different scope and assumption to our, due to the differences in the agents’ behavior, the missions that they have to perform and the different assumptions about the agents’ characteristics. 

The problem in \cite{alighanbari2005decentralized} refers to decentralized task assignment for a fleet of \uavs, being more similar to one in this current paper. The \uavs\ need to visit a set of targets, previously defined. The \uavs\ must decide which will visit each target, in a decentralized way. The authors in \cite{alighanbari2005decentralized} have demonstrated the drawback of using implicit coordination, which the central assignment algorithm is replicated in each UAV. When using implicit coordination all \uavs\ must have the same situational awareness. Then, a more robust extension is proposed. However their approach still assumes some degree of data synchronization among the \uavs. Even if there is a central entity, which creates the tasks, the proposed algorithm in this paper is purely decentralized, unlike \cite{alighanbari2005decentralized, tolmidis2013multi}. The central and even the agents do not need to have any kind of information about the others.
