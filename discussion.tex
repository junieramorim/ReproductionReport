The AL algorithm allows the \uavs\ to perform more tasks than Swarm-GAP by enabling the UAVs to select tasks while they still have resources to accomplish them.
Consequently, the elapsed time increases, since more tasks are performed. This is a positive feature of the proposed approach because the UAVs can use all available time. 
Furthermore, the number of exchanged messages among the \uavs, does not increase significantly when using AL. 
The AL algorithm also ensures that the token does not remain in the network forever (generating unnecessary communication) by informing the token when the \uav\ cannot perform any of the tasks that are still available in the token. However, as said, this availability check increases the runtime. 

When the tasks are sorted by tendency (SAL algorithm) it is more likely that each UAV uses its time to perform the most suitable tasks. Conversely, when using Swarm-GAP and AL, a \uav\ may end up using its time to perform other, less appropriate tasks. Thus, more tasks are performed than when using AL (without increasing the elapsed time and with higher quality). This provides an increase in the total reward.

In addition to allowing more tasks to be performed and giving preference to using resources in more appropriate tasks, the LAL algorithm provides a better workload balancing by the tasks selection limit. Thus, it improves performance further than AL and SAL. More tasks are performed in less time and the completed tasks quality increased even more. When using LAL, each UAV can choose just one task at a time, preventing some \uavs\ from selecting many tasks, becoming overloaded, while others \uavs\ remain idle.
% atendendo ao revisor 2:
However, the possibility of reaching an optimal allocation keeping some agents in idle is not ruled out. Such situations may happen. That is why the LAL method does not force an agent to choose a task: it can or cannot select a task, but only one at the most.

By construction, LAL causes more exchanged messages among the UAVs. Nevertheless, it is counter-balanced by the amount of performed tasks in less time. The LAL drawback is its runtime. It improves the allocation and reduces the tasks execution cost, but increases the runtime, since it runs more times than the other algorithms due to the selection limit. 
