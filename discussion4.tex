%Threats to validity
% scenario more dynamic
% other variables change

Basically, the methodology applied in this study was independent replications to collect results to find evidence that the original and extended algorithms are suitable to address the dynamic context. All assessment was done based on these results measured. Thus, for this process, the following threats to validity and corresponding mitigation strategies are presented in the following: 

\begin{itemize}
   \item \textbf{Conclusion validity}: The same code of the original study was used, but during the experiments different results were obtained, although they were in standard deviation compared with those obtained in the previous work, it indicates a variance possibility with changes in the experiment environment. These variations could be increased in case of code changes to implement dynamism. To minimize this threat, a set of tests and analysis were performed using a debug process to minimize undesirable responses due to a bad code manipulation after implementing the dynamism with changes in the number of agents. The results were analyzed to verify correlation and coherent with the new dynamic condition. As the acquired results can varies depending on the hardware configuration, each algorithm was executed a hundred of times to reduce the standard deviation. Furthermore, to comparison means, apparently equals to each other, a t-Test was used to identify these minor differences among the results obtained with the different proposed algorithms, after analyzing their similarity with a normal distribution.
   \item \textbf{Internal validity}: The performance of each algorithm depends on the execution environment. Since the hardware configuration was not the same and the tool version (NetLogo 5.3.1) uses some operating system libraries, there are a slight difference in the results obtained due to these different hardware and software configurations. To minimize this impact, a reproduction of the original study \cite{MAS07} was performed to have a suitable baseline for comparing the replications.
   \item \textbf{Construct validity}: All performed replications used the code obtained with the researcher of the original study, and the documentation lack became an issue in the moment of experiment execution. The modifications done to create the dynamic scenario could generate collateral effects difficult to be identified that could hide some influences and produce non-consistent results. To mitigate this threat, tests and analysis were performed using a debug process to minimize undesirable responses due to a bad code manipulation. Besides that, the same dependent variables were used, as well as units and measurement procedure from the original work, obtaining the results directly from the simulation tool (NetLogo 5.3.1).
   \item \textbf{External validity}: The extended algorithms here proposed were applied in a dynamic scenario where only the number of agents changes. Although the algorithms presented a good performance in this scenario, other elements can change in a real world. The number of tasks, type of tasks, status of the onboard sensors and other context issues that can occur in a real military operation. However, it was possible to approximate the original study, made in a static context, to something closer to the real world. This improvement allows an approach using Swarm-GAP intelligence in a context with a certain level of dynamism.
\end{itemize}

