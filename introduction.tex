The use of unmanned aerial vehicles (\uavs) to perform the so called dull, dirty and dangerous (3D) missions is becoming very common. There are many research focusing this theme, such as \cite{shirzadeh2017vision,sun2015route,kladis2011energy}. 
A special case is the use of UAVs for military purpose \cite{nonami2010autonomous}. New applications with multiple \uavs\ have been planned  \cite{zheng2004coevolving,UAVsmith2014autonomous,UAVsong2014persistent,UAVqu2015research}, in which \uavs\ cooperatively patrol perimeters, monitor areas of interest or escort convoys. Areas of difficult access, borders regions and critical infrastructure are examples of application scenarios that are easier monitored by groups of \uavs.

%Adicionado exemplos de sensores (para atender ao revisor 1)
Several kinds of sensors can be used for monitoring purposes, varying according to the situation. Examples of these sensors are: image sensors like RGB or thermal cameras, chemical sensors, radar sensors, among others. For instance, RGB cameras can be used in surveillance tasks to identify an object of interest while thermal cameras can be used in search and rescue operations, detection of fire spots or night vision. Each \uav\ can be equipped with one or more of these sensors, but they have limited resources such as time or energy (batteries or fuel that limit their endurance). In order to deploy an application in which a fleet of \uavs\ is designed to monitor a given area, these aspects have to be taken into account. If a massive usage of this type of system is considered, a centralized approach to allocate surveillance tasks to the \uavs\ does not scale \cite{alighanbari2005decentralized}.

In military operations it is common to have a central command unit that coordinates and delegates missions to be performed by military teams acting on the field. However, most commonly, the teams that receive these missions have autonomy to internally decide which members will perform the different parts of the mission. Observing this organization structure, this work focuses on teams of \uavs\, that must autonomously and cooperatively complete a mission assigned by the central command entity. A team of \uavs\ is seen as a group that receives a given mission, which contains a set of tasks, and internally has to take care of the division of it among its members.

This problem can be handled as a task allocation among agents, in which the \uavs\ are the agents and the mission is associated with a set of tasks. Many efforts have ever been made to solve the task allocation problem in several domains, and different approaches have been proposed, such as threshold-based \cite{ferreira2007swarm,scerri2005allocatingLADCOP,ferreira2010robocup,ikemoto2010adaptive} and market-based methods \cite{lemaire2004distributed,landen2010complex,ibri2012multi,tolmidis2013multi}. In \cite{scerri2005allocatingLADCOP}, for instance, a threshold-based algorithm was proposed to solve the task allocation in a rescue operation scenario. In \cite{landen2010complex}, an auction-based method was proposed to solve the multi-agent task allocation in the context of a multi-UAV system. Unlike the present problem, in \cite{scerri2005allocatingLADCOP,landen2010complex} the tasks are not sent by a central entity, but they are perceived by the agents in the environment.

Swarm intelligence is an appropriate alternative to deal with the multi-UAV task allocation problem in a decentralized way by using a threshold-based approach. Thus, each \uav\ can decide which tasks it will perform considering only local information, such as its location and resources status. This problem can be modeled using the generalized assignment problem (GAP). The GAP is known to be NP-Complete \cite{shmoys1993approximation}.
In the related literature, there is a heuristic method for task allocation based on swarm intelligence, called Swarm-GAP\cite{ferreira2007swarm} (see Section \ref{sec:background}), which allows agents to perform task allocation in an autonomous and decentralized way.  In this method, there is no central command unit that has knowledge of the set of tasks. Rather,  these tasks are perceived by the agents and they ``communicate'' (pass information about perceived tasks) to  other agents through a token-based communication protocol.

Swarm-GAP presents efficient results when the agents themselves perceive the tasks that need to be performed in the environment in which they act, create the tokens, and send them to other agents. Swarm-GAP works best when several tokens are created, each containing few tasks.
However, when a token contains many tasks, as is the case of tokens created by a central command unit, Swarm-GAP is less suitable because it cannot make an efficient use of agents' resources. The consequence is that many tasks are not selected by the agents, even if they have enough resources to execute them.

As mentioned before, the assumption of the existence of a central commander unit is reasonable: in  the case of military operations, the central entity's role that creates the tasks is of capital importance because this entity has a holistic view of the situation, which facilitates strategic planning. In general, when it comes to tasks that are delegated by a central entity, a token contains multiple tasks. Since the Swarm-GAP algorithm is not efficient for this situation, there is a need to  provide a more suitable solution.

Therefore, the contribution of this paper is the proposal of a new method with three algorithm variants, which aim to: (i) allow the agents use their resources to perform as many tasks as possible; (ii) avoid the agents assigning their resources to tasks that are not very suitable for them; and (iii) allow an efficient workload balance among the agents, that is, to prevent some agents from being overloaded with tasks while others remain idle.

The remainder of this paper is organized as follows. In Section \ref{sec:background}, the Swarm-GAP algorithm is briefly  presented. Section \ref{sec:problem} states the problem. The proposed solutions are described in Section \ref{sec:methods}. The description of the experimental setup is described in Section \ref{sec:experimental_setup}. The experiments and their results are presented and discussed in Section \ref{sec:experimental_result}. Section \ref{sec:rw} discusses related work. Section \ref{sec:conclusions} then makes concluding remarks and indicates future directions.