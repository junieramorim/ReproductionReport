In military operations the dynamism is a constant aspect presented in all scenarios and can bring a high level impact to the results and the operation itself. Mission and teams changes are some dynamic elements in this context and requires a reorganization and coordination to get an adaption to the new scenario. At this point, the use of Unmanned aerial vehicles (\uav) to perform missions are common. However, it is necessary to consider a limited quantity of resources, e.g. battery, fuel and sensors on board. Thereby, when the team of \uav has a certain level of autonomy, it needs to plan the execution of tasks in order to optimize the resources application.

This problem is defined as a decentralized task allocation among agents. A good strategy to be used is the Swarm-GAP \cite{MOEA07}\cite{MAS07}, where the problem is solved in a decentralized way and basically based on local information that each agent has. Thus, each \uav can decide which tasks will be performed by it. However the swarm intelligence can have issues during the task allocation process caused by specific characteristics of some scenarios, e.g. the information exchange among \uavs can get a loop. Based on this, the original study \cite{MAS07} proposed three adaptations of Swarm-GAP algorithm in order to mitigate the issue previously cited.

The scenario used to validate that proposal was based on a central command that sends the mission. This mission is composed by a set of tasks that will be performed by the \uav team. The transmission of these information is done through a token based protocol. When a specific \uav chooses a task and allocate it to itself, it sends this information to the next agent of the team. Thus, the next agent will know which tasks are available to be executed. The proposal is a little change in the way of how these tokens travel through the \uavs.

Basically, the main aspect of the swarm strategy is to improve the autonomy of the system to decide the execution strategy. However there is a concern with the quality level. This quality is defined by the maximization of tasks performed and minimization of the execution time. Other aspects of quality can be considered by the system with a minor adaptation.

The three variants of swarm-GAP algorithm presented in \cite{MAS07} are: a) Allocation Loop (AL) where a control list of visited agents is used and the token runs while there is task available. To avoid the infinite loop, the agent registers if it has any resource available to perform a new task before resend the token. Only the agent with free resources will receive the token. b) Sorting and Allocation Loop (SAL) sorts the list of tasks based on the tendency of execution by the \uav. This avoids the first agent gets all tasks instead of them that are more suitable and produces better results. c) Limit and Allocation Loop (LAL) increases the SAL behavior and defines a task selection limit per \uav in each round to avoid a greedy strategy.

The contribution of the original paper was the proposal of a new method with those algorithm variants, which aim to: (i) allow the agents use their resources to perform as many tasks as possible; (ii) avoid the agents assigning their resources to tasks that are not very suitable for them; and (iii) allow an efficient workload balance among the agents, that is, to prevent some agents from being overloaded with tasks while others remain idle. However, there are no dynamic aspect considered and nothing changes after the \uavs receive the mission thought the token.

The purpose of this work is present the results obtained with the replication and reproduction of the experiment presented by \cite{MAS07}. The original experiment (\cite{MAS07}) presents the application of an heuristic method to solve the allocation tasks problem that combines a swarm intelligence and a generalized assignment problem (GAP) modelling, resulting in a Swarm-GAP algorithm \cite{ferreira2007swarm}. Furthermore, it presented three more variations of Swarm-GAP, listed above, to solve the allocation tasks problem in a scenario formed by a set of agents.

This work will consider reproducible research (RR) as a concept different from replication. These concepts are presented by \cite{exp02} and the basic difference among them is the fact of the replication changes some experiment elements. The reproduction comes to confirm all evidences and the solution proposed by the original work, in another hand the replication changes some elements of the experiment, e.g. participants and control variables.

Defined in \cite{exp03} the independent replication was applied to the original experiment. According to this definition, experiments were performed with changes in some variables in order to explore an wider context than the original one.  The dynamism was included to the system. 

***Para ajustar

The remainder of this paper is organized as follows. In Section \ref{sec:background}, the concepts involved during the replication is briefly  presented. Section \ref{sec:problem} shows the main objective of the replication. The scenarios and conditions description of the experiments in Section \ref{sec:methods}. The reproduction results in Section \ref{sec:experimental_setup}. The independent replication and their results are presented and discussed in Section \ref{sec:experimental_result}. Section \ref{sec:rw} discusses the results analysis. Section \ref{sec:conclusions} shows a brief conclusion and suggests future studies.